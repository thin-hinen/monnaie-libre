\documentclass{report}
\usepackage[french]{babel}
\usepackage[T1]{fontenc}
\usepackage[utf8]{inputenc}

\begin{document}
Producteev est une gestionnaire de tâches disponibles sous Android, iOS, Mac et par navigateur. L'inscription et l'utilisation sont gratuites.

\section{Réseau}
Producteev est basé sur un principe de «réseau». Chaque personne appartient à un «réseau» et peut ajouter d'autres utilisateur à ces «réseaux». On peut ensuite créer des projets dans un réseau, et y ajouter des personnes du réseau. 

\section{Projet}
\subsection{Tâches}
Un projet est une succession de tâches, définit par les utilisateurs (ou administrateurs, selon les droits définis). Ces tâches sont représentés par un nom (courte description de la tâche) et on peut leur définir une date de fin (répétable), assigner des personnes membre du projet (qui devront accomplir la tâche), définir des sous-tâche, ajouter des labels (voir partie Labels) ou encore laisser un message. 

\subsection{Activités}
Producteev permet aux utilisateurs de suivre l'évolution d'un projet grâce aux activités. Ces activités, qui sont par exemple la création d'une tâche ou son accomplissement, sont affichés avec la date ainsi que la personne l'ayant effectuée.

\section{Labels}
Les labels sont des mots-clés. Ils permettent de classer des tâches selon des critères (on peut imaginer par exemple un label modèle, un label vue et un label contrôle pour un projet basé sur le principe MVC). On peut sélectionner un de ces labels pour afficher toutes les tâches dans tous les projets du réseau en rapport avec ce label. Ils permettent ainsi de retrouver rapidement des tâches en lien les unes avec les autres.

\end{document}