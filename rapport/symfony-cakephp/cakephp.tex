
%mvc : http://book.cakephp.org/2.0/en/cakephp-overview/understanding-model-view-controller.html
%tutoriel de blog : http://book.cakephp.org/2.0/en/tutorials-and-examples/blog/blog.html

%http://book.cakephp.org/2.0/en/tutorials-and-examples.html
\begin{frame}
\begin{center}
\includegraphics[scale=0.40]{img/noidea.jpg}

Je dis une connerie $\Rightarrow$ interrompez moi !
\end{center}
\end{frame}

\section{Structure des fichiers}

\begin{frame}{/}
\begin{description}
  \item [app] Le dossier ou le code de l'application CakePHP est placé (notre dossier de travail)
  \item [lib] Le dossier où le code de CakePHP est placé
  \item [vendors] Autres bibliothèques PHP
  \item [plugins]
  \item [.htaccess]
  \item [index.php]
  \item [README]
\end{description}
\end{frame}

\begin{frame}{/app/}
\begin{description}
  \item [Config] Les fichiers de configuration y sont placés
  \item [Console] Les commandes pour la console
  \item [Controller] Les controlleurs du motif MVC
  \item [Lib] Classes ou bibliothèques CakePHP d'origine interne
  \item [Locale] Internationalisation
  \item [Model] Le modèle du motif MVC
  \item [Plugin] Des plugins...
  \item [Test] Contient des cas de test (pour tester le bon fonctionnement de l'application)
  \item [tmp] Utilisé par CakePHP pour stocker des données temporairement
  \item [Vendor] Classes ou bibliothèques CakePHP d'origine tierce
  \item [View] La vue du motif MVC
  \item [webroot] Utilisé comme racine pour les documents produits, les fichiers CSS, JavaScript ou image doivent être placés ici
\end{description}
\end{frame}

\end{document}

